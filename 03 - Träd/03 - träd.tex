% The format (A5) is selected to facilitate reading on small
% devices and should NOT be changed. 
\documentclass[a5paper,10pt,oneside]{article}

% The package babel is loaded for Swdish with Swedish hyphenation,
% replaces "Contents" with "Innehållsförteckning, "References"
% with "Litteraturförteckning", etc.
\usepackage[swedish]{babel}

\usepackage[T1]{fontenc}

% The package "inputenc" lets us specify what character encoding
% has been used to save the .tex file. If your computer runs
% Linux, the encoding is probably "utf8" by default, while under
% Windows the default will probably be "latin1" The wrong
% character encoding may give strange signs instead of "å", "ä"
% and "ö" or may result in compilation errors.

%\usepackage[latin1]{inputenc} % Probably right if you use Windows
\usepackage[utf8]{inputenc}  % Probably right if you use Linux

% The packages listed below are optional and can be removed if you
% don't use them 
\usepackage{graphicx} 
\usepackage{cite}
\usepackage{url}
\usepackage{ifthen}
\usepackage{listings}	

% These two lines set up options for the listings package and
% can be removed if you don't use it, or changed if you, e.g, 
% use another language than Java. 
% For more information about the listings package see:
% ftp://ftp.tex.ac.uk/tex-archive/macros/latex/contrib/listings/listings.pdf
\def \lstlistingname {Kodexempel}
\lstset{language=Java,tabsize=3,numbers=left,frame=L,floatplacement=hbtp}


\usepackage{ifpdf}
\ifpdf
	\usepackage[hidelinks]{hyperref}
\else
	\usepackage{url}
\fi


% Change NR and TITLE below to appropriate values
\title{Tema NR: 3 TRÄD}

% Write the name and user namn for all participants in the group here.
% Separate persons with \and
\author{Oscar Törnquist \url{osta3589} \and Emil Rosell \url{emro9957}}




\begin{document}

\maketitle

% Here the actual report starts. Everything from here to the start of the
% bibilography should, of course, be removed before you start writing your 
% own text.

\section*{Muntafrågor}
\begin{enumerate}
\item Förklara hur removefunktionen fungerar och vilka problem som kan uppstå, förklara även hur dessa problem kan förebyggas.
\item Vad är det som skiljer sig mellan enkelrotation och dubbelrotation, varför är det viktigt för balanserade binära sökträd?
\item Ge exempel på vad som skiljer sig mellan Splay-träd och AVL-träd och kontrastera sorternas för och nackdelar samt användningsområden.
\item Traverseringsordningarna inorder, preorder och postorder fungerar på olika sätt, förklara hur och varför det är nödvändigt att använda sig av olika ordningar i olika situationer.
\end{enumerate}




\end{document}
