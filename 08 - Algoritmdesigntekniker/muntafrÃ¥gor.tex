% The format (A5) is selected to facilitate reading on small
% devices and should NOT be changed. 
\documentclass[a5paper,10pt,oneside]{article}

% The package babel is loaded for Swdish with Swedish hyphenation,
% replaces "Contents" with "Innehållsförteckning, "References"
% with "Litteraturförteckning", etc.
\usepackage[swedish]{babel}

\usepackage[T1]{fontenc}

% The package "inputenc" lets us specify what character encoding
% has been used to save the .tex file. If your computer runs
% Linux, the encoding is probably "utf8" by default, while under
% Windows the default will probably be "latin1" The wrong
% character encoding may give strange signs instead of "å", "ä"
% and "ö" or may result in compilation errors.

%\usepackage[latin1]{inputenc} % Probably right if you use Windows
\usepackage[utf8]{inputenc}  % Probably right if you use Linux

% The packages listed below are optional and can be removed if you
% don't use them 
\usepackage{graphicx} 
\usepackage{cite}
\usepackage{url}
\usepackage{ifthen}
\usepackage{listings}	

% These two lines set up options for the listings package and
% can be removed if you don't use it, or changed if you, e.g, 
% use another language than Java. 
% For more information about the listings package see:
% ftp://ftp.tex.ac.uk/tex-archive/macros/latex/contrib/listings/listings.pdf

\usepackage{ifpdf}
\ifpdf
	\usepackage[hidelinks]{hyperref}
\else
	\usepackage{url}
\fi


% Change NR and TITLE below to appropriate values
\title{Tema NR: 8 Algoritmdesigntekniker}

% Write the name and user namn for all participants in the group here.
% Separate persons with \and
\author{Oscar Törnquist \url{osta3589} \and Emil Rosell \url{emro9957}}

\begin{document}

\maketitle

\section*{Muntafrågor}
\begin{enumerate}
\item Förklara varför den rekursiva Fibonacci algoritmen inte passar till kategorin Divide and Conquer. För VG, förklara hur en sådan algoritm fungerar och ge exempel.

\item Förklara skillnaden mellan on-line packing och off-line packing när det gäller binpacking. För VG, kontrastera binpacking med exempelvis Prim's eller Kruskall's algoritmen.

\item Vad är en skiplist och varför kan skiplists vara såpass effektiva? För VG, visa bra förståelse för algoritmer som är slumpbaserade och förklara varför slump används i skiplists.
\item
Vad menas med en girig algoritm? För VG, förklara hur Huffmans algoritm fungerar och vad som gör att det anses vara en girig algoritm.
\item
Vad är dynamisk programmering? För VG, förklara hur \textit{Optimal Binary Search Tree} och \textit{All-Pairs Shortest Path} fungerar . 
\item
Förklara vad en girig algoritm är samt ge några exempel. För VG, förklara varför giriga algoritmer ibland kan ge felberäkningar, och varför de ändå används?

\end{enumerate}

\end{document}
